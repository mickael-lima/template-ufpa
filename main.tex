\documentclass[
	% -- opções da classe memoir --
    article,
	12pt,				% tamanho da fonte
	a4paper,			% tamanho do papel.
	% -- opções da classe abntex2 --
    % chapter=TITLE,	% títulos de capítulos convertidos em letras maiúsculas
    section=TITLE,		% títulos de seções convertidos em letras maiúsculas
	subsection=TITLE,	% títulos de subseções convertidos em letras maiúsculas
	subsubsection=TITLE,% títulos de subsubseções convertidos em letras maiúsculas
    o
    % -- opções do pacote babel --
	english,			% idioma adicional para hifenização
	french,			% idioma adicional para hifenização
	spanish,			% idioma adicional para hifenização
	brazil				% o último idioma é o principal do documento
]{abntex2}

\usepackage{graphicx}
\usepackage{mathptmx} % Fonte times new romam no texto
\renewcommand{\ABNTEXchapterfont}{\rmfamily\bfseries} % Fonte times new romam em negrito nos itens
\usepackage{microtype} % para melhorias de justificação
\usepackage{pdfpages}

% ---
% Pacotes de citações
% ---
\usepackage[brazilian,hyperpageref]{backref} % Paginas com as citações na bibl
\usepackage[alf]{abntex2cite} % Citações padrão ABNT

% Configurações de aparência do PDF final
\usepackage{color}				% Controle das cores
% alterando o aspecto da cor azul
\definecolor{blue}{RGB}{41,5,195}

% informações do PDF
\makeatletter
\hypersetup{
     	%pagebackref=true,
		pdftitle={\@title},
		pdfauthor={\@author},
    	pdfsubject={\imprimirpreambulo},
	    pdfcreator={LaTeX e abnTeX2},
		pdfkeywords={abnt}{latex}{abntex}{abntex2}{trabalho acadêmico},
		colorlinks=true,       		% false: boxed links; true: colored links
    	linkcolor=blue,          	% color of internal links
    	citecolor=blue,        		% color of links to bibliography
    	filecolor=magenta,      		% color of file links
		urlcolor=blue,
		bookmarksdepth=4
}
\makeatother

\makeindex

% --- Capa gerada pelo comando \imprimircapa

\renewcommand{\imprimircapa}{%
  \begin{capa}%

    % header
    \center

    \vspace{-1cm}
    % --- brasão da UFPA
    \begin{figure}[h]
      \centering
      \includegraphics[width=2.5cm]{img/Brasao_UFPA}
    \end{figure}

    \ABNTEXchapterfont\bfseries\large\imprimirinstituicao\\
    \vfill

    % centro + título + nomes
    \begin{center}
      \ABNTEXchapterfont\bfseries\large\imprimirtitulo\\
      \vspace{4cm}
      \ABNTEXchapterfont\bfseries\large\imprimirautor
    \end{center}

    % footer
    \vfill
    \ABNTEXchapterfont\large\imprimirlocal\\
    \ABNTEXchapterfont\large\imprimirdata
    \vspace*{1cm}
  \end{capa}
}

% --- folha de rosto
\makeatletter
\renewcommand{\folhaderostocontent}{
  \begin{center}

    % --- header ---
    {\ABNTEXchapterfont\large\imprimirautor}\par % \par necessário para corrigir bug de formatação
    \vspace*{\fill}\vspace*{\fill}
    {\ABNTEXchapterfont\bfseries\large\imprimirtitulo}

    \vfill

    % --- centro ---
    \abntex@ifnotempty{\imprimirpreambulo}{%
      \hspace{.45\textwidth}
      \begin{minipage}{.5\textwidth}
        \SingleSpacing
        \imprimirpreambulo
      \end{minipage}%
      \vspace*{4cm}
    }%

    {\large\imprimirorientadorRotulo~\imprimirorientador\par}
    \abntex@ifnotempty{\imprimircoorientador}{%
      {\large\imprimircoorientadorRotulo~\imprimircoorientador}%
    }%

    % --- footer ---
    \vspace*{\fill}
    {\large\imprimirlocal}
    \par
    {\large\imprimirdata}
    \vspace*{1cm}
  \end{center}
}
\makeatother


% --- configurações gerais para o trabalho ---

% --- capa ---
\titulo{Titulo teste de trabalho academico}
\autor{Mickael Lima Henry}
\local{Belém $-$ PA}

% Caso precise de mais um campo por qualquer motivo, siga
% a organização abaixo e funcionará como esperado
\instituicao{%
  UNIVERSIDADE FEDERAL DO PARÁ
  \par
  FACULDADE DE ALGUMA COISA
  \par
  CURSO DE ALGUMA ÁREA
}

\data{2024}

% --- contra-capa ---
\preambulo{Algum preambulo interessante o suficiente, caso necessite}
\orientador[Orientador(a): ]{Prof quaisquer}

\begin{document}

% -- configurações gerais pós-documento
\selectlanguage{brazil}
\frenchspacing

\pretextual
\imprimircapa
\imprimirfolhaderosto

% --- alguns exemplos de elementos pré textuais possiveis
\include{content/dedicatoria.tex}

% --- resumo em português ---
\begin{resumo}
  Resumo em português caso o trabalho necessite desse tipo de estrutura (obrigatórios em TCC)

  \vspace{\onelineskip}
  \noindent
  \textbf{Palavras-chaves}: latex. abntex. editoração de texto.
\end{resumo}

\newpage

\begin{resumo}[Abstract]
\begin{otherlanguage*}{english}
  You can do it in other language too.\par

  \vspace{\onelineskip}
  \noindent

  \textbf{Keywords}: latex. abntex. template.
\end{otherlanguage*}
\end{resumo}


\pdfbookmark[0]{\contentsname}{toc}
\tableofcontents*
\cleardoublepage

% --- conteúdo (coloque os .tex de conteúdo aqui) ---
%
% por ex: crie um arquivo intro.tex na pasta content/ para armazenar a introdução do trabalho e, depois que terminar
% de editá-lo, coloque
%
% \section{Introdução}

Esse arquivo exemplifica a organização por meio desse pseudo-template. Para realizar citações, utilize o comando \texttt{cite} e edite o arquivo \texttt{bibliografia.bib} como instruido no repositório desse template, um exemplo de citação \cite{abibliaemingles}

%
% abaixo de \pagestyle para incluir esse .tex, isso permitirá
% com que os arquivos fiquem mais organizados
\textual
\pagestyle{simple}

\section{Introdução}

Esse arquivo exemplifica a organização por meio desse pseudo-template. Para realizar citações, utilize o comando \texttt{cite} e edite o arquivo \texttt{bibliografia.bib} como instruido no repositório desse template, um exemplo de citação \cite{abibliaemingles}



% --- pós-conteúdo / final do projeto (ex: bibliografia de referências)
\newpage
\postextual

\bibliography{bibliografia}
\end{document}
